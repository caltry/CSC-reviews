% Author: David Larsen <dcl9934@cs.rit.edu>
\documentclass[11pt]{article}
\usepackage{fullpage}
\usepackage{listings}
\usepackage{needspace}
\usepackage{color}
\usepackage{ifthen}

\lstset{ %
basicstyle=\footnotesize,       % the size of the fonts that are used for the code
numbers=left,                   % where to put the line-numbers
numberstyle=\footnotesize,      % the size of the fonts that are used for the line-numbers
stepnumber=1,                   % the step between two line-numbers. If it's 1 each line will be numbered
numbersep=5pt,                  % how far the line-numbers are from the code
showspaces=false,               % show spaces adding particular underscores
showstringspaces=false,         % underline spaces within strings
tabsize=4,		                % sets default tabsize to 4 spaces
language=Python
}

\ifthenelse{\isundefined{\isAnswerKey}}
{
    \newenvironment{answer}{\lstset{basicstyle=\large}\color{white}}{}
}
{
    \newenvironment{answer}{\lstset{basicstyle=\large}}{}
}


\begin{document}
\noindent{\Large CS-241 Midterm Exam Review \hfill Winter, 2011-2}

\begin{enumerate}
    \item\label{reverse()} Write a function which reverses a string (e.g. ``Don't
        get sick'' becomes ``kcis teg t'noD'').

\begin{answer}
\begin{lstlisting}
def reverse( string ):
    if len( string ) < 2:
        return string
    else:
        return reverse( string[1:] ) + string[0]
\end{lstlisting}
\end{answer}

    \item Write a function that takes in a string and returns a string that
          reverses the letters in each word, but keeps the word ordering the
          same.  (e.g. reverse\_words(``I wear a Stetson now -- Stetsons are
          cool'') returns ``I raew a nostetS won -- snostetS era looc''). You
          may use the function reverse() from question \#\ref{reverse()} in
          your solution.

\begin{answer}
\begin{lstlisting}
def reverse_words( string ):
    split_string = string.split()
    return reverse( split_string[0] ) + reverse_words( split_string[1:] )
\end{lstlisting}
\begin{center}{\LARGE OR}\end{center}
\begin{lstlisting}
def reverse_words( string):
    str = ""
    for word in string.split():
        str += reverse( word )
    return str
\end{lstlisting}
\end{answer}

    \item Write a function that takes in a file name, and returns the average
        size of a word eg. a file containing:\\

        lots of work\\
        no rest for midterms\\
        sad for you\\

        has an average length of: 3.6

\begin{answer}
\begin{lstlisting}
def reverse_file( filename ):
    characters = 0
    words = 0
    for line in open(filename):
        for word in line.split():
            words += 1
            characters += len(word)
    return characters/words
\end{lstlisting}
\end{answer}

    \item Given the following function bar:
\needspace{15\baselineskip}
\begin{lstlisting}
def bar( baz ):
    if len( baz ) < 2:
        return baz
    pivot = baz[0]
    smaller = list()
    pivot_list = list()
    larger = list()
    for element in baz:
        if element == pivot:
            pivot_list.append(element)
        elif element < pivot:
            smaller.append(element)
        else:
            larger.append(element)
    return bar(smaller) + pivot_list + bar(larger)
\end{lstlisting}

        \begin{enumerate}
            \item Perform a substitution trace on bar( [4,3,6,32,9] ).
\begin{answer}
\begin{lstlisting}
bar([3]) + [4] + bar([6,32,9])
bar([3]) + [4] + bar([]) + [6] + bar([32,9])
bar([3]) + [4] + bar([]) + [6] + bar([9]) + [32] + bar([])
[3] + [4] + [] + [6] + [] + [9] + [32]
[3,4,6,9,32]
\end{lstlisting}
\end{answer}

            \item What’s the base case for `bar'?

                \begin{answer}
                When len(baz) $<$ 2, so [] and [n].
                \end{answer}
            \item What does `bar' do?

                \begin{answer}
                `bar' sorts the list of numbers (it’s actually quicksort).
                \end{answer}

            \item How would you test this function?

                \begin{answer}
                Empty list
                1 element list
                2 element list
                3 element list
                sorted list
                unsorted list
                reverse-sorted list
                \end{answer}

        \end{enumerate}

    \item Write a function that takes in a list of numbers and returns the sum
        of all of those numbers.
        \begin{enumerate}
            \item Recursively.
\begin{answer}
\begin{lstlisting}
def sum( numbers ):
    if len(numbers) == 0:
        return 0
    else
        return numbers[0] + sum( numbers[1:] )
\end{lstlisting}
\end{answer}

            \item Iteratively
\begin{answer}
\begin{lstlisting}
def sum( numbers ):
    index = 0
    sum = 0
    while index < len(numbers):
        sum += numbers[index]
        index += 1
    return sum
\end{lstlisting}
\end{answer}

            \item How would you test this function?
                \begin{answer}
                \begin{itemize}
                    \item Empty list
                    \item 1 element list
                    \item multi-element list
                \end{itemize}
                \end{answer}
        \end{enumerate}

\end{enumerate}
\end{document}

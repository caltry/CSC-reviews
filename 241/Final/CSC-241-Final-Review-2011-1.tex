% Author: David Larsen <dcl9934@cs.rit.edu>
\documentclass[11pt]{article}
\usepackage{fullpage}
\usepackage{listings}

\lstset{ %
basicstyle=\footnotesize,       % the size of the fonts that are used for the code
numbers=left,                   % where to put the line-numbers
numberstyle=\footnotesize,      % the size of the fonts that are used for the line-numbers
stepnumber=1,                   % the step between two line-numbers. If it's 1 each line will be numbered
numbersep=5pt,                  % how far the line-numbers are from the code
showspaces=false,               % show spaces adding particular underscores
showstringspaces=false,         % underline spaces within strings
tabsize=4,		                % sets default tabsize to 4 spaces
language=Java
}


\begin{document}
\noindent{\Large CS-241 Final Exam Review \hfill \today}

\begin{enumerate}
    \item Write a function that recursively sums the elements of a list of
          integers.
          \vspace{2.5in}

    \item Given the linked list: $1 \rightarrow 2 \rightarrow 3$
        \begin{enumerate}
            \item 1 points to 2 and 2 points to 3. What does 3 point to?
                \vspace{0.5in}
            \item Draw out the linked list structure and add a 5 to the end.
                \vspace{1.25in}
            \item Write some pseudo-code for appending data to the end of a list.
                \vspace{3.5in}
            \pagebreak
            \item Looking at the list from (b), you notice that we forgot to
                  add in a 4. What is a procedure for doing this? (There are
                  many possibilities)
                \vspace{2.5in}
        \end{enumerate}
    \item Write a Python function, given a sorted linked list, returns a Python
    list of all of the elements which are in the list more than once. (eg.
    $1\rightarrow2\rightarrow2\rightarrow3\rightarrow3\rightarrow3$ would return [2,3,3] )
        \vspace{3in}
    \item If your solution to (3) was iterative, write it recursively; if it
    was recursive, write it iteratively.
        \pagebreak
    \item Write a function which reverses a linked list.
        \vspace{4in}
    \item If we have a balanced binary search tree (its height is as small as
    possible) containing 'n' nodes.
        \begin{enumerate}
            \item What is its height?
                \vspace{.25in}
            \item How much time would it take to traverse to any of the leaf
            nodes of this tree.
                \vspace{.25in}
            \item What's the worst case search time for an (unbalanced) search
            tree?
                \pagebreak
        \end{enumerate}
    \item Write a function that determines if a given element is in a binary search tree.
        \vspace{3in}
    \item Write a function which counts the number of leaf nodes in a given binary tree.
        \pagebreak
    \item Create a binary search tree, drawing the whole tree
        again for each element inserted from the list: [1,2,8,3,5].
        \vspace{6in}
    \item Describe the behavior of a greedy algorithm.
        \pagebreak

    \item What does the following code snippet do?
        \begin{verbatim}
        def theRentIsTooDamnHigh( n ):
            if len(n) <= 1:
                return n
            return theRentIsTooDamnHigh(n[1:]) + n[0]
        \end{verbatim}
            \vspace{1in}
        Show a substitution trace of theRentIsTooDamnHigh( ``meat'' )
            \vspace{2in}
    \item Write a function called ``removeOpt( string )'' that:
        \begin{itemize}
            \item Takes a string as an argument
            \item returns a string with all substrings in [brackets] removed.
        \end{itemize}
        For example:    
            removeOpt("This [is] [a] [nice] string")
            returns ``this   string'' 
        \pagebreak
    
    \item Write a function that returns a dictionary identifying how often each
    letter appears in a string. Do not include frequencies for letters that
    were not in the string. For example running your function on ``a can''
    would return the dictionary {'a': 2, 'c': 1, 'n': 1}.
        \vspace{3in}
    
    \item Draw the output of this turtle code
        \begin{verbatim}
from turtle import *
def todayIsTheDay( length, n ):
    if( n <= 0 ):
        return
    forward( length )
    left( 90 )
    todayIsTheDay( length//2, n-1 )
    right( 180 )
    todayIsTheDay( length//2, n-1 )
    left( 90 )
    back( length )

todayIsTheDay( 100, 3 )
        \end{verbatim}

        \pagebreak
    
    \item Write a function which reads in a string and returns a list of all of
    the words in the string, broken up by spaces, without using the
    string.split() function.
        \vspace{5in}

    \item Below is python code for a function that performs an insertion sort
    and prints ``data'' after each iteration of the for-loop.
\begin{verbatim}
def insertion_sort( data ):
    for marker in range( 1, len( data ) ):
        val = data[marker]
        i = marker
        while i > 0 and data[i-1] > val:
            data[i] = data[i-1]
            i -= 1
        data[i] = val
        print( data )
\end{verbatim}
        Write out what the function will print for the input list: [3,2,7,1].
        \pagebreak

    \item Write the body for the function ``wordCounter'', which opens a text
    file and records the number of occurrences of each word in the file. You
    should use a dictionary to track the occurrences, and return the completed
    dictionary from the function.
        \vspace{3.5in}

    \item For each of the operations below, give the big-O time complexity for
    both a Linked List and an Array List.
        \begin{enumerate}
            \item Insert at the beginning\vspace{0.2in}
            \item Remove at the beginning\vspace{0.2in}
            \item Insert at the end\vspace{0.2in}
            \item Remove at the end\vspace{0.2in}
            \item Access the last element\vspace{0.2in}
        \end{enumerate}
    When is it better to use an array list over a linked list?
\end{enumerate}
\end{document}

Old stuff follows.
\noindent{\Large CS-243 \hfill Week 10: Review}
\begin{enumerate}
	% Patterns
	\item When you're writing an ordered Collection, how can you determine if one
		generic object is supposed to come before another?
		\vspace{1in}
	\item Create a BufferedReader that reads its input from the keyboard.
		\vspace{.3in}
	\item Now create a BufferedReader as above, but catch all of the exceptions.
		\vspace{3in}
	\item When can one object access another object's private variables?
		\vspace{.5in}	% When they are from the same class
	\item When an object ``decorates'' another object, what happens?
		\vspace{.75in}
	
\end{enumerate}

\documentclass[11pt]{article}
\usepackage{fullpage}
\usepackage{listings}
\usepackage{needspace}
\usepackage{color}
\usepackage{ifthen}
\usepackage{pgf}
\usepackage{tikz}
\usetikzlibrary{arrows,automata}
\usepackage{amsmath}

\lstset{ %
basicstyle=\footnotesize,       % the size of the fonts that are used for the code
numbers=left,                   % where to put the line-numbers
stepnumber=1,                   % the step between two line-numbers. If it's 1 each line will be numbered
numbersep=5pt,                  % how far the line-numbers are from the code
showspaces=false,               % show spaces adding particular underscores
showstringspaces=false,         % underline spaces within strings
tabsize=4,		                % sets default tabsize to 4 spaces
language=Java
}

\ifthenelse{\isundefined{\isAnswerKey}}
{
    \newenvironment{answer}{\large\lstset{basicstyle=\large}\color{white}}{}
}
{
    \newenvironment{answer}{\large\lstset{basicstyle=\large}\color{red}}{}
}


\author{Computer Science Community}
\title{CS-243 Final Exam Review}
\date{Spring, 2011-3}

\makeatletter
\let\thetitle\@title
\let\theauthor\@author
\let\thedate\@date
\makeatother

\begin{document}
\noindent{\Large \thetitle \hfill \thedate}

\begin{enumerate}

\item {\bf Nick's Heavy Threads} Nick operates a store in Marketview Mall which
      has poor lighting, blasts black metal and sells jeans. Only one pair of
      jeans are available to purchase at a time, though there are more
      available in the back. If a size is out that you don't want, you must
      wait for someone else to purchase the jeans. Nick's only employee, Hank,
      sits in a chair and stares at people angrily until someone makes a
      purchase, when he replaces the jeans with the same model of a random
      size.
    
\begin{lstlisting}
public class NicksHeavyThreads
{
    Jeans awesomeJeans;
    public static void main(String... args)
    {
      MeanWorker.start();
      for(int i=0; i < 10; ++i)
      {
          LameCustomer.start();
      }
    }
}
\end{lstlisting}

    \begin{enumerate}
    \item What basic threading model can deal with this?
        \begin{enumerate}
        \item Banker's algorithm    \begin{answer}~\end{answer}
        \item Producer/Consumer     \begin{answer}True\end{answer}
        \item Y Combinator          \begin{answer}~\end{answer}
        \item Pessimistic locking   \begin{answer}~\end{answer}
        \end{enumerate}
    
    \item Write Hank's class. Recall that he must replace jeans as soon as they
          are sold.

        \begin{answer}
\begin{lstlisting}
public class MeanWorker
{
    public void run()
    {
        synchronized (awesomeJeans) {
            if( awesomeJeans == 0 )
            {
                awesomeJeans++;
            }
        }
    }
\end{lstlisting}
        \end{answer}
    \end{enumerate}


\item{\bf B+Trees} A B+Tree is a tree made up of arrays.

\begin{figure}
\caption{A partially drawn $\textrm{B}^+$-Tree}
\includegraphics[width=5in]{b_plus_tree_dot.pdf}
\end{figure}

    \begin{enumerate}
    \item We need classes to represent the nodes of the tree. Implement these
    classes so that they use generic types and all of their members are public.

\begin{answer}
\begin{lstlisting}
public interface Node<K,V> {}

public class InternalNode<K,V> implements BTreeNode<K,V>
{
    K[] keys;
    Node<K,V>[] children;
}

public class LeafNode<K,V> implements BTreeNode<K,V>
{
    K keys[];
    V values[];
}
\end{lstlisting}
\end{answer}

    \end{enumerate}

\end{enumerate}

\end{document}

Topics for this exam: (from the ZJB!)
Oh yeah!  We're moving topics around on the schedule (which won't impact
the exam, obviously), de-emphasizing some of the design patterns, and
(the big one) putting in threads.  Still not sure what detail we'll have
with threads, but about one week's worth.

Here is what will go on the syllabus:

Week #  Schedule
------  ---------------------------
1        Intro to Java
2        Classes (scope and equals)
3        Interfaces
4        Inheritance (equals pt2)
5        JCF, O-O design
6        Iterators, Comparators, and intro to Swing
7        Event driven programming
8        Exceptions, I/O
9        Threads
10       Design Patterns


\documentclass[11pt]{article}
\usepackage{fullpage}
\usepackage{listings}
\usepackage{needspace}
\usepackage{color}
\usepackage{ifthen}
\usepackage{pgf}
\usepackage{tikz}
\usetikzlibrary{arrows,automata}
\usepackage{amsmath}
\usepackage{url}
\usepackage{framed}

\lstset{ %
basicstyle=\footnotesize,       % the size of the fonts that are used for the code
numbers=left,                   % where to put the line-numbers
stepnumber=1,                   % the step between two line-numbers. If it's 1 each line will be numbered
numbersep=5pt,                  % how far the line-numbers are from the code
showspaces=false,               % show spaces adding particular underscores
showstringspaces=false,         % underline spaces within strings
tabsize=4,		                % sets default tabsize to 4 spaces
language=Java
}

\ifthenelse{\isundefined{\isAnswerKey}}
{
    \newenvironment{answer}{\large\lstset{basicstyle=\large}\color{white}}{}
}
{
    \newenvironment{answer}{\large\lstset{basicstyle=\large}\color{red}}{}
}


\author{Computer Science Community}
\title{CS-243 Final Exam Review}
\date{Spring, 2011-3}

\makeatletter
\let\thetitle\@title
\let\theauthor\@author
\let\thedate\@date
\makeatother

\begin{document}
\noindent{\Large \thetitle \hfill \thedate}\\
\noindent{\small Presented by the RIT Computer Science Community \hfill
\url{http://csc.cs.rit.edu}}

\begin{enumerate}

\item {\bf Nick's Heavy Threads} Nick operates a store in Marketview Mall which
      has poor lighting, blasts black metal and sells jeans. Only one pair of
      jeans are available to purchase at a time, though there are more
      available in the back. If a size is out that you don't want, you must
      wait for someone else to purchase the jeans. Nick's only employee, Hank,
      sits in a chair and stares at people angrily until someone makes a
      purchase, when he replaces the jeans with the same model of a random
      size.
    
\begin{lstlisting}
public class NicksHeavyThreads
{
    Jeans awesomeJeans;
    public static void main(String... args)
    {
      MeanWorker.start();
      for(int i=0; i < 10; ++i)
      {
          LameCustomer.start();
      }
    }
}
\end{lstlisting}

    \begin{enumerate}
    \item What basic threading model can deal with this?
        \begin{enumerate}
        \item Banker's algorithm    \begin{answer}~\end{answer}
        \item Producer/Consumer     \begin{answer}True\end{answer}
        \item Y Combinator          \begin{answer}~\end{answer}
        \item Pessimistic locking   \begin{answer}~\end{answer}
        \end{enumerate}
    
    \item Write Hank's class. Recall that he must replace jeans as soon as they
          are sold.

        \begin{answer}
\begin{lstlisting}
public class MeanWorker
{
    public void run()
    {
        synchronized (awesomeJeans) {
            if( awesomeJeans == 0 )
            {
                awesomeJeans++;
            }
        }
    }
\end{lstlisting}
        \end{answer}
    \end{enumerate}


\item{\bf $\textrm{B}^+$Trees} A $\textrm{B}^+$ tree is a data structure that
can be used to map from keys to values. Figure \ref{b-tree} shows a tree that
maps from ints to Strings. Note that there are 4 slots for values, even though
there are only 3 keys. This is because the arrays of keys in the top two rows
specify a range: [3,5,7] are less than 10, so they go into the leftmost
reference, while [15,20,\_] are all $\geq 20$ and $<35$.

The bottom row of nodes have a direct correspondence between a key and a value.
And each level of nodes is connected together as a linked list.

\begin{figure}
\caption{A partially drawn $\textrm{B}^+$-Tree of strings, keyed on integers}
\label{b-tree}
\center
\includegraphics[width=5in]{b_plus_tree_dot.pdf}
\end{figure}

    \begin{enumerate}
    \item What Java Collections Framework interface should a
    $\textrm{B}^+$-Tree be able to implement?

        \begin{answer}
        Java.util.Map$<$K,V$>$
        \end{answer}

    \item We need to implement two different kinds of nodes for the tree. Why
    do we need to do this? What is different between the two types of nodes?

        \begin{answer}
        Some nodes (the leaves) point to data values. We also have internal
        nodes which point to other nodes (both internal nodes and leaves).
        \end{answer}

    \item We need classes to represent the nodes of the tree. Implement these
    classes so that they use generic types and all of their members are package
    private.

\begin{answer}
\begin{lstlisting}
public interface Node<K,V> {}

public class InternalNode<K,V> implements BTreeNode<K,V>
{
    K[] keys;
    Node<K,V>[] children;
    InternalNode<K,V> next;
}

public class LeafNode<K,V> implements BTreeNode<K,V>
{
    K keys[];
    V values[];
    LeafNode<K,V> next;
}
\end{lstlisting}
\end{answer}

    \end{enumerate}

\item What is the output when LookAtDatMagic's main is executed?
\begin{lstlisting}
public class HeySteve{
    public int bananza(int in) throws NewException{
        if ( in == 7 ){
            throw new NewException("HeySteve, cut that out!");
        }
        return in;
    }
}

public class NewException extends Exception{
    public NewException(String message){
        super(message);
    }

}
           
public class WakaWaka{
    public String BeachBash(Object a, Object b) throws NewException{
        if ( a.equals(b) ){
            throw new NewException("It's a Beach-bash! WakaWaka!");
        }
        return "Da-nanananan";
    }
}

public class LookAtDatMagic{
    public void magic() throws NewException{
        int maraca = 5;
        try{
            HeySteve steve = new HeySteve();
            maraca = steve.bananza(7);
        }catch(NewException e){
            System.out.println(e.getMessage());
        }finally{
            WakaWaka waka = new WakaWaka();
            System.out.println(waka.BeachBash(maraca, 5));
        }
    }

    public static void main(String[] args){
        try{
            LookAtDatMagic ladm = new LookAtDatMagic();
            ladm.magic();
        }catch(NewException e){
            System.out.println(e.getMessage());
        }
    }
}


\end{lstlisting}
\begin{answer}
HeySteve, cut that out!


It's a Beach-bash! WakaWaka!
\end{answer}

\item Briefly describe the difference between classes and objects.\\
\begin{answer}
    A class is a construct which describes the properties and methods of an object. It can be thought of as a mold or template. An object is an 
instantiation of a class. That is, a specific item created from the class "mold".
\end{answer}

\item Briefly describe the difference (for objects) between a.equals(b), a==b, compare(a,b), and a.compareTo(b).
\begin{answer}
    \begin{itemize}
        \item a.equals(b) Compares objects for equality. Automatically defined by java, but must be overwritten
to provide intelligent behavior returns a boolean
    \item a == b  Checks references (if the two objects are the SAME object). Can also be used to check if a is null  
    \item a.compareTo(b) returns an int indicating if the a is less than (-1), equal to(0), or greater than b(+1). Indicates natural ordering 
    \item compare(a,b) returns -1 if a $<$ b, 0 if a $=$ b, +1 if a $>$ b. comp1.equals(comp2) implies that sgn(comp1.compare(o1, o2))==sgn(comp2.compare(o1, o2)) for every object reference o1 and o2.
    \end{itemize}
\end{answer}

\item Name the design pattern used in the following snippet of code.
\begin{lstlisting}
public class Car
{
	private String make;
	private String model;
	private int mileage;

	public Car(String make, String model, int mileage)
	{
		this.make=make;
		this.model=model;
		this.mileage=mileage;
	}
	public static Car makeCar(String str)
	{
		String arr[] = str.split(" ");
		return new Car(arr[0], arr[1], Integer.parseInteger(arr[2]));
	}

	public static void main(String[] args)
	{
		Car myCar = Car.makeCar("Toyota Camry 200000");
	}
}
\end{lstlisting}

\begin{answer}
The \emph{Factory} design pattern is used.
\end{answer}

\item 
    \begin{enumerate} 
    \item What is the Java Collections Framework?
    \begin{answer}
        a unified set of interfaces, algorithms and concrete implementaions 
    provided by java to support collections of objects.
    \end{answer}
    \item Assume the follwing line of code is given:
    \begin{lstlisting}
        Collection<Integer> t = new ArrayList<Integer>();
    \end{lstlisting}
    What then is wrong with the following? Correct any errors:
    \begin{lstlisting}
        for( int i = 0; i < 20; i++){
            t.add(i);
        }
        for( int i=0; i <  t.size(); i++ ){
            System.out.println(t.get(i));
        }
    \end{lstlisting}
    \begin{answer}
    Collection does not support get(i). The better solution 
    \begin{lstlisting}
        for( int i = 0; i < 20; i++){
            t.add(i);
        }
        for( Integer i : t ){
            System.out.println(i);
        }

    \end{lstlisting}
    \end{answer}
    \end{enumerate}
    

\item Given the following class, Write the code necessary for the main to run.
\begin{lstlisting}
public class UFO{
    private Collection<Probable> toProbe;
    private int aggression;

    public UFO(int agg){
        toProbe = new ArrayList<Probable>();
        aggression = agg;
    }

    public void probeEverything(){
        for( Probable p : toProbe ){
            p.probe();
        }
    }

    public void sendHome(){
        for( Probable p: toProbe ){
            p.returnHome();
        }
    }

    public void abduct(Collection<Probable> potentialAbductees){
        for( Probable p : potentialAbductees ){
            if( p.getMentalResolve() < aggression){
                toProbe.add(p);
            }
        }
    }

    public static void main(String[] args){
        UFO ufo = new UFO(1199999999);

        field.add(new Cow("Bessie"));
        //10 = mentalResolve
        Human cleatus = new RedNeck("Cleatus", 10));
        
        // 50 == mentalResolve and 100 == academicRespect
        Human brown = new Professor("Brown", 50, 100));         
        
        field.add(cleatus);
        field.add(brown);

        ufo.abduct(field);
        ufo.probeEverything();
        ufo.sendHome();
    }
}
\end{lstlisting}
\begin{answer}
\begin{lstlisting}
public interface Probable{
    public void probe();

    public int getMentalResolve();

    public void returnHome();
}
public class Cow implements Probable{
    private String name;

    public Cow(String name){
        this.name = name;
    }

    public int getMentalResolve(){
        return 10;
    }

    public void probe(){
       System.out.println("moo.");
    }

    public void returnHome(){
        System.out.println(name + " is unimpressed.");
    }
}
public abstract class Human implements Probable{
    private int mentalResolve;
    private String name;

    public Human(String nm, int res){
        name = nm;
        mentalResolve = res;
    }

    public abstract void probe();

    public abstract void returnHome();

    public String getName(){
        return name;
    }

    public int getMentalResolve(){
        return mentalResolve;
    }
}

public class Professor extends Human{
    private int academicRespect;

    public Professor(String name, int mental, int resp){
        super(name, mental);
        academicRespect = resp;
    }

    public void probe(){
        System.out.println("\"What an exhilarating day for mankind!\", "
                                  + "shouts Prof. " + getName());
        System.out.println("\"Oh my!\", exclaims Prof. " + getName());
    }

    public void returnHome(){
        adjustAcademicCareerToFocusOnAbductions();
    }

    public void adjustAcademicCareerToFocusOnAbductions(){
        academicRespect = 0;
    }
}
public class RedNeck extends Human{
    public RedNeck(String nm, int res){
        super(nm, res);
    }

    public void probe(){
        System.out.println("\"Wut y'all doin wit dat der deevise?\", "
                                           + "inquires " + getName() );
        System.out.println("\"GAAAAH!!\", exlaims "+ getName() );
    }

    public void returnHome(){
        beGuestOnJerrySpringerToTalkAboutTheProbing();
    }

    public void beGuestOnJerrySpringerToTalkAboutTheProbing(){
        System.out.println("I's tellin ya Jerry!");
    }

}

\end{lstlisting}
\end{answer}
\end{enumerate}



\vfill
\begin{framed}
How was our review? Please let us know by filling out our survey at\\
\url{http://clipboard.rit.edu}, survey id: FCF9EA34
\end{framed}


\end{document}

Topics for this exam: (from the ZJB!)
Oh yeah!  We're moving topics around on the schedule (which won't impact
the exam, obviously), de-emphasizing some of the design patterns, and
(the big one) putting in threads.  Still not sure what detail we'll have
with threads, but about one week's worth.

Here is what will go on the syllabus:

Week #  Schedule
------  ---------------------------
1        Intro to Java
2        Classes (scope and equals)
3        Interfaces
4        Inheritance (equals pt2)
5        JCF, O-O design
6        Iterators, Comparators, and intro to Swing
7        Event driven programming
8        Exceptions, I/O
9        Threads
10       Design Patterns


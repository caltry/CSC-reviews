\documentclass[11pt]{article}
\usepackage{fullpage}
\usepackage{listings}
\usepackage{needspace}
\usepackage{color}
\usepackage{ifthen}
\usepackage{pgf}
\usepackage{tikz}
\usetikzlibrary{arrows,automata}
\usepackage{amsmath}

\lstset{ %
basicstyle=\footnotesize,       % the size of the fonts that are used for the code
numbers=left,                   % where to put the line-numbers
stepnumber=1,                   % the step between two line-numbers. If it's 1 each line will be numbered
numbersep=5pt,                  % how far the line-numbers are from the code
showspaces=false,               % show spaces adding particular underscores
showstringspaces=false,         % underline spaces within strings
tabsize=4,		                % sets default tabsize to 4 spaces
language=Python
}

\ifthenelse{\isundefined{\isAnswerKey}}
{
    \newenvironment{answer}{\large\lstset{basicstyle=\large}\color{white}}{}
}
{
    \newenvironment{answer}{\large\lstset{basicstyle=\large}\color{red}}{}
}


\author{Computer Science Community}
\title{CS-242 Final Exam Review}
\date{Winter, 2011-2}

\makeatletter
\let\thetitle\@title
\let\theauthor\@author
\let\thedate\@date
\makeatother

\begin{document}
\noindent{\Large \thetitle \hfill \thedate}

\begin{enumerate}
\section*{Backtracking}
\section*{Hashing and Hash Tables}
\section*{Sorting}

\item\label{qsort-worst-case} What kind of data causes Quicksort's worst-case
      time complexity?

      \begin{answer}
      Data that is (nearly) sorted or is sorted in reverse order.
      \end{answer}

\item What causes Quicksort to run so slowly on the input you describe in
      question \ref{qsort-worst-case}?

    \begin{answer}
    {\huge FIXME}
    \end{answer}

\item In Quicksort, why do we select a random pivot value, rather than always
      pivoting on the first element?

      \begin{answer}
      With real-world data, we're more likely to encounter ordered or
      semi-ordered data than randomised data. This makes it more likely for us
      run into Quicksort's worst-case time complexity. We run into this bad
      time complexity if we select pivots which are near the lowest or highest
      values.

      Selecting a random value to pivot on helps us encounter the average case
      evens out the distribution of ordered and unordered data. Even if we're
      getting in sorted data, if we select pivots randomly, we should be able
      to end up with average time complexity.
      \end{answer}

\section*{Heaps and Heapsort}

\section*{Dynamic Programming}

\item When we trace all of the function calls being made while computing the
      n'th number in the Fibonacci sequence from a solution given earlier in
      the quarter, we notice that there is a lot of extra work being done.

      \marginpar{\small\em You can safely ignore the things mentioning
      `ftrace': those are just there to make drawing the call graph easy.}
      \lstinputlisting{naive_fibonacci.py}

\begin{verbatim}
>>> fib(4)
    0 fib( (4,) {} )
    1 |  fib( (3,) {} )
    2 |  |  fib( (2,) {} )
    3 |  |  |  fib( (1,) {} )               # fib(1) called
    3 |  |  |  fib( (1,) {} ) returns: 1
    3 |  |  |  fib( (0,) {} )
    3 |  |  |  fib( (0,) {} ) returns: 0
    2 |  |  fib( (2,) {} ) returns: 1
    2 |  |  fib( (1,) {} )                  # fib(1) called again
    2 |  |  fib( (1,) {} ) returns: 1
    1 |  fib( (3,) {} ) returns: 2
    1 |  fib( (2,) {} )
    2 |  |  fib( (1,) {} )                  # fib(1) called once more
    2 |  |  fib( (1,) {} ) returns: 1
    2 |  |  fib( (0,) {} )
    2 |  |  fib( (0,) {} ) returns: 0
    1 |  fib( (2,) {} ) returns: 1
    0 fib( (4,) {} ) returns: 3
\end{verbatim}

      Write an implementation of `fib' which will not do duplicate work.

    \begin{answer}
    \marginpar{\small\em Note the approach to default assignment for the table:
    if table was set to [0,1] in parameter list, we would be modifying it
    globally whenever we appended to it.}
    \lstinputlisting{memoized_fibonacci.py}
\begin{verbatim}
>>> fib(4)
    0 fib( (4,) {} )
    1 |  fib( (3, [0, 1]) {} )
    2 |  |  fib( (2, [0, 1]) {} )
    3 |  |  |  fib( (1, [0, 1]) {} )
    3 |  |  |  fib( (1, [0, 1]) {} ) returns: 1
    3 |  |  |  fib( (0, [0, 1]) {} )
    3 |  |  |  fib( (0, [0, 1]) {} ) returns: 0
    2 |  |  fib( (2, [0, 1]) {} ) returns: 1
    2 |  |  fib( (1, [0, 1, 1]) {} )
    2 |  |  fib( (1, [0, 1, 1]) {} ) returns: 1
    1 |  fib( (3, [0, 1]) {} ) returns: 2
    1 |  fib( (2, [0, 1, 1, 2]) {} )
    1 |  fib( (2, [0, 1, 1, 2]) {} ) returns: 1
    0 fib( (4,) {} ) returns: 3
\end{verbatim}
    \end{answer}

\end{enumerate}

\end{document}

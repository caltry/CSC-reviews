\documentclass[11pt]{article}
\usepackage{fullpage}
\usepackage{listings}
\usepackage{needspace}
\usepackage{color}
\usepackage{ifthen}
\usepackage{pgf}
\usepackage{tikz}
\usetikzlibrary{arrows,automata}
\usepackage{amsmath}

\lstset{ %
basicstyle=\footnotesize,       % the size of the fonts that are used for the code
numbers=left,                   % where to put the line-numbers
stepnumber=1,                   % the step between two line-numbers. If it's 1 each line will be numbered
numbersep=5pt,                  % how far the line-numbers are from the code
showspaces=false,               % show spaces adding particular underscores
showstringspaces=false,         % underline spaces within strings
tabsize=4,		                % sets default tabsize to 4 spaces
language=Python
}

\ifthenelse{\isundefined{\isAnswerKey}}
{
    \newenvironment{answer}{\large\lstset{basicstyle=\large}\color{white}}{}
}
{
    \newenvironment{answer}{\large\lstset{basicstyle=\large}\color{red}}{}
}


\author{Computer Science Community}
\title{CS-242 Final Exam Review}
\date{Winter, 2011-2}

\makeatletter
\let\thetitle\@title
\let\theauthor\@author
\let\thedate\@date
\makeatother

\begin{document}
\noindent{\Large \thetitle \hfill \thedate}

\begin{enumerate}
\section*{Backtracking}
\section*{Hashing and Hash Tables}
\section*{Sorting}

\item\label{qsort-worst-case} What kind of data causes Quicksort's worst-case
      time complexity?

      \begin{answer}
      Data that is (nearly) sorted or is sorted in reverse order.
      \end{answer}

\item What causes Quicksort to run so slowly on the input you describe in
      question \ref{qsort-worst-case}?

    \begin{answer}
    {\huge FIXME}
    \end{answer}

\item In Quicksort, why do we select a random pivot value, rather than always
      pivoting on the first element?

      \begin{answer}
      With real-world data, we're more likely to encounter ordered or
      semi-ordered data than randomised data. This makes it more likely for us
      run into Quicksort's worst-case time complexity. We run into this bad
      time complexity if we select pivots which are near the lowest or highest
      values.

      Selecting a random value to pivot on helps us encounter the average case
      evens out the distribution of ordered and unordered data. Even if we're
      getting in sorted data, if we select pivots randomly, we should be able
      to end up with average time complexity.
      \end{answer}

\section*{Heaps and Heapsort}

\section*{Dynamic Programming}

\item We're trying to find out how durable eggs are by dropping them off of a
      36 story building. We want to find the break point for the eggs with as
      few drops as possible (because this building doesn't have an elevator,
      and we hate walking up stairs!), but we only have 2 eggs. We want to
      find a general solution for the minimum number of drops given 1 or more
      eggs and a building with any number of stories.
      
    \begin{enumerate}
    \item We'll be solving this problem with dynamic programming. We'll need
          to use a two dimensional table which uses the number of test eggs
          available and the number of stories we haven't tested on. What
          should be the value that we store at that location?
          
        \begin{answer} The value at that index will be the number of steps
        taken to get to that state.

          \[\textrm{weight}( \textrm{eggs}, \textrm{stories left} ) =
          \textrm{minimal number \# of steps to get to this point}\]
        \end{answer}

    \item Write the function which will fill in a single cell of the table.
    {\\\Huge TODO: Place generated table here}

        \begin{answer}
        \begin{align*}
        W(e,s) = 1 + min\{ & max(W(e-1,1-1),W(e,s-1)),\\
                           & max(W(e-1,2-1),W(e,s-2)),\\
                           & \cdots,\\
                           & max(W(e-1,(s-1)-1),W(e,s-(s-1))),\\
                           & max(W(e-1,s-1),W(e,s-s))\}
        \end{align*}

        {\Huge TODO: Also implement in Python.}

        \end{answer}
    
    \item What location in the table has our final answer?

    \end{enumerate}

\end{enumerate}

\end{document}

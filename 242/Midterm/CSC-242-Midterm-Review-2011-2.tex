% Author: David Larsen <dcl9934@cs.rit.edu>
\documentclass[11pt]{article}
\usepackage{fullpage}
\usepackage{listings}
\usepackage{needspace}
\usepackage{color}
\usepackage{ifthen}

\lstset{ %
basicstyle=\footnotesize,       % the size of the fonts that are used for the code
numbers=left,                   % where to put the line-numbers
stepnumber=1,                   % the step between two line-numbers. If it's 1 each line will be numbered
numbersep=5pt,                  % how far the line-numbers are from the code
showspaces=false,               % show spaces adding particular underscores
showstringspaces=false,         % underline spaces within strings
tabsize=4,		                % sets default tabsize to 4 spaces
language=Python
}

\ifthenelse{\isundefined{\isAnswerKey}}
{
    \newenvironment{answer}{\large\lstset{basicstyle=\large}\color{white}}{}
}
{
    \newenvironment{answer}{\large\lstset{basicstyle=\large}}{}
}


\author{Computer Science Community}
\title{CS-242 Midterm Exam Review \---- Winter 2011-2}

\begin{document}
\noindent{\Large CS-242 Midterm Exam Review \hfill Winter, 2011-2}

\begin{enumerate}
\section*{Stacks and Queues}
    \item What data structure is important for doing a depth-first traversal?

        \begin{answer}
        A stack.
        \end{answer}

    \item If you want to find the shortest path to a node on a graph, what type
        of traversal would be best to use? Why?

        \begin{answer}
        A breadth-first traversal will {\em always} find a path with the
        minimum number of hops. This happens because we evaluate all of the
        nodes at the same distance away from the start point at the same time.
        We can say that the first time we see the node we want, we will have
        taken the shortest path to reach it.
        \end{answer}

    \item If we wanted to implement a stack as a linked list, what linked-list
        operations would correspond to push() and pop()?

        \begin{answer}
        push( stack, element ) $\rightarrow$ insertFront( linked-list, element )\\
        pop( stack ) $\rightarrow$ removeFront( linked-list )
        \end{answer}

\section*{Graph Searching}
    \item foobar
\section*{Backtracking}
    \item foobar
   
\end{enumerate}

\end{document}
